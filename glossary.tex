% glossaries can be configured by defining the \glsopts command before including it.
% \newcommand{\glsopts}{acronyms,nolist,<other options>}
\providecommand{\glsopts}{acronyms,nolist}
\usepackage[\glsopts]{glossaries}
\makeglossaries

\newacronym{anomaly_detection}{AD}{anomaly detection}
\newacronym{artificial_intelligence}{AI}{artificial intelligence}

\newacronym{base_band_unit}{BBU}{base band unit}

\newacronym{central_unit}{CU}{Central Unit}
\newacronym{cloud_ran}{C-RAN}{cloud RAN}
\newacronym{comercial_off_the_shelf}{COTS}{Commercial off-the-shelf}
\newacronym{computing_resource}{CR}{computing resource}
\newacronym{control_plane}{CP}{control plane}

\newacronym{data_plane}{DP}{data plane}
\newacronym{devops}{DevOps}{development and operations}
\newacronym{disagregated_ran_combination}{DRC}{disaggregated RAN combination}
\newacronym{distributed_unit}{DU}{Distributed Unit}

\newacronym{general_purpose_processor}{GPP}{general purpose processor}
\newacronym{graph_neural_network}{GNN}{graph neural network}

\newacronym{long_term_evolution}{LTE}{long term evolution}

\newacronym{management_and_network_orchestration}{MANO}{management and network orchestration}
\newacronym{medium_access_contorl}{MAC}{medium access control}
\newacronym{message_passing_graph_neural_network}{MPGNN}{Message-Passing Graph Neural Network}
\newacronym{multiple_input_multiple_output}{MIMO}{multiple input/multiple output}

\newacronym{near_real_time_ran_intelligent_controller}{nRT-RIC}{near real-time RAN intelligent controller}
\newacronym{network_function}{NF}{network function}
\newacronym{network_function_virtualization}{NFV}{network function virtualization}
\newacronym{network_function_virtualization_infrastructure}{NFVI}{network function virtualization infrastructure}
\newacronym{network_function_virtualization_orchestrator}{NFVO}{network function virtualization orchestrator}
\newacronym{network_service_orchestrator}{NSO}{network service orchestrator}
\newacronym{next_generation_ran}{NG-RAN}{next generation radio access network}
\newacronym{non_real_time_ran_intelligent_controller}{non-RT-RIC}{non real-time RAN intelligent controller}

\newacronym{open_air_interface}{OAI}{OpenAirInterface}
\newacronym{open_radio_access_network}{O-RAN}{Open RAN}

\newacronym{packet_data_convergence_protocol}{PDCP}{packet data convergence protocol}
\newacronym{phy}{PHY}{Physical Layer}

\newacronym{quality_of_service}{QoS}{quality of service}
\newacronym{quality_of_service_predictor}{QP}{QoS predictor}

\newacronym{radio_access_network}{RAN}{radio access network}
\newacronym{radio_frequency}{RF}{radio frequency}
\newacronym{radio_link_control}{RLC}{radio link control}
\newacronym{radio_resource_control}{RRC}{radio resource control}
\newacronym{radio_resource_management}{RRM}{radio resource management}
\newacronym{radio_unit}{RU}{Radio Unit}
\newacronym{ran_intelligent_controller}{RIC}{RAN intelligent controller}
\newacronym{resource_orchestrator}{RO}{Resource Orchestrator}
\newacronym{ric_messaging_router}{RMR}{RIC messaging router}

\newacronym{service_function_chain}{SFC}{service function chain}
\newacronym{service_management_and_orchestration}{SMO}{service management and orchestration}
\newacronym{software_defined_network}{SDN}{software defined network}

\newacronym{total_cost_of_ownership}{TCO}{Total cost of ownership}
\newacronym{traffic_steering}{TS}{traffic steering}

\newacronym{user_device}{UD}{user device}
\newacronym{user_plane}{UP}{user plane}

\newacronym{virtual_network_function}{VNF}{virtual network function}
\newacronym{virtualized_central_unit}{vCU}{virtualized central unit}
\newacronym{virtualized_distributed_unit}{vDU}{virtualized distributed unit}
\newacronym{virtualized_infrastructure_manager}{VIM}{virtualized infrastructure manager}
\newacronym{virtualized_network_functions_manager}{VNFM}{virtual network function manager}
\newacronym{virtualized_ngran}{vNG-RAN}{virtualized next-generation RAN}
\newacronym{virtualized_ran}{vRAN}{virtualized radio access network}
\newacronym{virtualized_radio_unit}{vRU}{virtualized radio unit}


%Glossary
\newglossaryentry{application_plane_g}{
    name={application plane},
    description={The collection of applications and services that program network behavior}}
    
\newglossaryentry{control_plane_g}{
    name={control plane},
    description={The collection of functions responsible for controlling one or more network devices. The CP instructs network devices with respect to how to process and forward packets. The control plane interacts primarily with the forwarding plane and, to a lesser extent, with the operational plane.}}
    
\newglossaryentry{forwarding_plane_g}{
    name={forwarding plane},
    description={The collection of resources across all network devices responsible for forwarding traffic.}}
    
\newglossaryentry{management_plane_g}{
    name={management plane},
    description={The collection of functions responsible for monitoring, configuring, and maintaining one or more network devices or parts of network devices. The management plane is mostly related to the operational plane (it is related less to the forwarding plane)}}
    
\newglossaryentry{multiple_input_multiple_output_g}{
    name={multiple input/multiple output},
    description={Technique for sending and receiving multiple data signals from the same radio signal}}

\newglossaryentry{network_function_virtualization_g}{
    name={Network Function Virtualization},
    description={The replacement of network appliance hardware with virtual machines}}
    
\newglossaryentry{nfv_infrastructure_g}{
    name={network function virtualization infrastructure},
    description={Totality of all hardware and software components that build up the environment in which VNFs are deployed}}
    
\newglossaryentry{nfv_mano_g}{
    name={network function virtualization management and orchestration},
    description={Functions collectively provided by NFVO, VNFM, and VIM}}
    
\newglossaryentry{nfv_orchestrator_g}{
    name={network function virtualization orchestrator},
    description={Functional block that manages the Network Service (NS) life cycle and coordinates the management of NS life cycle, VNF life cycle (supported by the VNFM) and NFVI resources (supported by the VIM) to ensure an optimized allocation of the necessary resources and connectivity}}
